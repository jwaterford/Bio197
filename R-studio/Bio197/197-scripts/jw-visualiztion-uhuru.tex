% Options for packages loaded elsewhere
\PassOptionsToPackage{unicode}{hyperref}
\PassOptionsToPackage{hyphens}{url}
%
\documentclass[
]{article}
\usepackage{amsmath,amssymb}
\usepackage{lmodern}
\usepackage{iftex}
\ifPDFTeX
  \usepackage[T1]{fontenc}
  \usepackage[utf8]{inputenc}
  \usepackage{textcomp} % provide euro and other symbols
\else % if luatex or xetex
  \usepackage{unicode-math}
  \defaultfontfeatures{Scale=MatchLowercase}
  \defaultfontfeatures[\rmfamily]{Ligatures=TeX,Scale=1}
\fi
% Use upquote if available, for straight quotes in verbatim environments
\IfFileExists{upquote.sty}{\usepackage{upquote}}{}
\IfFileExists{microtype.sty}{% use microtype if available
  \usepackage[]{microtype}
  \UseMicrotypeSet[protrusion]{basicmath} % disable protrusion for tt fonts
}{}
\makeatletter
\@ifundefined{KOMAClassName}{% if non-KOMA class
  \IfFileExists{parskip.sty}{%
    \usepackage{parskip}
  }{% else
    \setlength{\parindent}{0pt}
    \setlength{\parskip}{6pt plus 2pt minus 1pt}}
}{% if KOMA class
  \KOMAoptions{parskip=half}}
\makeatother
\usepackage{xcolor}
\usepackage[margin=1in]{geometry}
\usepackage{color}
\usepackage{fancyvrb}
\newcommand{\VerbBar}{|}
\newcommand{\VERB}{\Verb[commandchars=\\\{\}]}
\DefineVerbatimEnvironment{Highlighting}{Verbatim}{commandchars=\\\{\}}
% Add ',fontsize=\small' for more characters per line
\usepackage{framed}
\definecolor{shadecolor}{RGB}{248,248,248}
\newenvironment{Shaded}{\begin{snugshade}}{\end{snugshade}}
\newcommand{\AlertTok}[1]{\textcolor[rgb]{0.94,0.16,0.16}{#1}}
\newcommand{\AnnotationTok}[1]{\textcolor[rgb]{0.56,0.35,0.01}{\textbf{\textit{#1}}}}
\newcommand{\AttributeTok}[1]{\textcolor[rgb]{0.77,0.63,0.00}{#1}}
\newcommand{\BaseNTok}[1]{\textcolor[rgb]{0.00,0.00,0.81}{#1}}
\newcommand{\BuiltInTok}[1]{#1}
\newcommand{\CharTok}[1]{\textcolor[rgb]{0.31,0.60,0.02}{#1}}
\newcommand{\CommentTok}[1]{\textcolor[rgb]{0.56,0.35,0.01}{\textit{#1}}}
\newcommand{\CommentVarTok}[1]{\textcolor[rgb]{0.56,0.35,0.01}{\textbf{\textit{#1}}}}
\newcommand{\ConstantTok}[1]{\textcolor[rgb]{0.00,0.00,0.00}{#1}}
\newcommand{\ControlFlowTok}[1]{\textcolor[rgb]{0.13,0.29,0.53}{\textbf{#1}}}
\newcommand{\DataTypeTok}[1]{\textcolor[rgb]{0.13,0.29,0.53}{#1}}
\newcommand{\DecValTok}[1]{\textcolor[rgb]{0.00,0.00,0.81}{#1}}
\newcommand{\DocumentationTok}[1]{\textcolor[rgb]{0.56,0.35,0.01}{\textbf{\textit{#1}}}}
\newcommand{\ErrorTok}[1]{\textcolor[rgb]{0.64,0.00,0.00}{\textbf{#1}}}
\newcommand{\ExtensionTok}[1]{#1}
\newcommand{\FloatTok}[1]{\textcolor[rgb]{0.00,0.00,0.81}{#1}}
\newcommand{\FunctionTok}[1]{\textcolor[rgb]{0.00,0.00,0.00}{#1}}
\newcommand{\ImportTok}[1]{#1}
\newcommand{\InformationTok}[1]{\textcolor[rgb]{0.56,0.35,0.01}{\textbf{\textit{#1}}}}
\newcommand{\KeywordTok}[1]{\textcolor[rgb]{0.13,0.29,0.53}{\textbf{#1}}}
\newcommand{\NormalTok}[1]{#1}
\newcommand{\OperatorTok}[1]{\textcolor[rgb]{0.81,0.36,0.00}{\textbf{#1}}}
\newcommand{\OtherTok}[1]{\textcolor[rgb]{0.56,0.35,0.01}{#1}}
\newcommand{\PreprocessorTok}[1]{\textcolor[rgb]{0.56,0.35,0.01}{\textit{#1}}}
\newcommand{\RegionMarkerTok}[1]{#1}
\newcommand{\SpecialCharTok}[1]{\textcolor[rgb]{0.00,0.00,0.00}{#1}}
\newcommand{\SpecialStringTok}[1]{\textcolor[rgb]{0.31,0.60,0.02}{#1}}
\newcommand{\StringTok}[1]{\textcolor[rgb]{0.31,0.60,0.02}{#1}}
\newcommand{\VariableTok}[1]{\textcolor[rgb]{0.00,0.00,0.00}{#1}}
\newcommand{\VerbatimStringTok}[1]{\textcolor[rgb]{0.31,0.60,0.02}{#1}}
\newcommand{\WarningTok}[1]{\textcolor[rgb]{0.56,0.35,0.01}{\textbf{\textit{#1}}}}
\usepackage{graphicx}
\makeatletter
\def\maxwidth{\ifdim\Gin@nat@width>\linewidth\linewidth\else\Gin@nat@width\fi}
\def\maxheight{\ifdim\Gin@nat@height>\textheight\textheight\else\Gin@nat@height\fi}
\makeatother
% Scale images if necessary, so that they will not overflow the page
% margins by default, and it is still possible to overwrite the defaults
% using explicit options in \includegraphics[width, height, ...]{}
\setkeys{Gin}{width=\maxwidth,height=\maxheight,keepaspectratio}
% Set default figure placement to htbp
\makeatletter
\def\fps@figure{htbp}
\makeatother
\setlength{\emergencystretch}{3em} % prevent overfull lines
\providecommand{\tightlist}{%
  \setlength{\itemsep}{0pt}\setlength{\parskip}{0pt}}
\setcounter{secnumdepth}{-\maxdimen} % remove section numbering
\ifLuaTeX
  \usepackage{selnolig}  % disable illegal ligatures
\fi
\IfFileExists{bookmark.sty}{\usepackage{bookmark}}{\usepackage{hyperref}}
\IfFileExists{xurl.sty}{\usepackage{xurl}}{} % add URL line breaks if available
\urlstyle{same} % disable monospaced font for URLs
\hypersetup{
  pdftitle={Uhuru Data Set Visualization},
  pdfauthor={James Waterford},
  hidelinks,
  pdfcreator={LaTeX via pandoc}}

\title{Uhuru Data Set Visualization}
\author{James Waterford}
\date{2023-02-23}

\begin{document}
\maketitle

\begin{figure}
\centering
\includegraphics{https://live.staticflickr.com/7185/6922756579_49638c4951_b.jpg}
\caption{Acacia}
\end{figure}

The Working Directory inside this Rmarkdown \emph{chunk} is listed
below:

\begin{Shaded}
\begin{Highlighting}[]
\FunctionTok{getwd}\NormalTok{()}
\end{Highlighting}
\end{Shaded}

\begin{verbatim}
## [1] "/Users/slimjims/Desktop/master/R-studio/Bio197/197-scripts"
\end{verbatim}

UHURU studies the effect of herbivores on the plants in Kenya.

This data set focuses on the effects of different treatments on Acacia
trees. These \emph{treaments} are different areas of Kenya, were
predation levels (and species) are diversified.

We will first call the file, based on the current directory. Then we
will use \texttt{read.table()} to access the file.

Using \texttt{eval\ =\ TRUE} displays the data.

\begin{Shaded}
\begin{Highlighting}[]
\DocumentationTok{\#\# Provide file using a relative path \#\#}
\NormalTok{acacia\_csv }\OtherTok{\textless{}{-}}\NormalTok{ (}\StringTok{"../197{-}raw\_storage/ACACIA\_DREPANOLOBIUM\_SURVEY.txt"}\NormalTok{)}
\DocumentationTok{\#\# add \textasciigrave{}na.strings = "dead"}
\NormalTok{acacia }\OtherTok{\textless{}{-}} \FunctionTok{read.table}\NormalTok{(acacia\_csv, }\AttributeTok{header =} \ConstantTok{TRUE}\NormalTok{, }\AttributeTok{sep =} \StringTok{"}\SpecialCharTok{\textbackslash{}t}\StringTok{"}\NormalTok{, }\AttributeTok{fill =} \ConstantTok{TRUE}\NormalTok{)}
\end{Highlighting}
\end{Shaded}

What can we quickly find out from this data set?

There are a handful of commands that can help us out. Some personal
favorites are:

\texttt{head} - read the first 10 lines from the first 10 columns
\texttt{summary} - does simple mathematical calculations on your data
set \texttt{str} - displays 10 values from each column, with their data
type \texttt{View} - creates a table in another tab \texttt{class} -

\begin{Shaded}
\begin{Highlighting}[]
\FunctionTok{head}\NormalTok{(acacia)}
\end{Highlighting}
\end{Shaded}

\begin{verbatim}
##   SURVEY YEAR  SITE BLOCK TREATMENT    PLOT   ID HEIGHT AXIS1 AXIS2 CIRC
## 1      1 2012 SOUTH     1     TOTAL S1TOTAL  581   2.25  2.75  2.15   20
## 2      1 2012 SOUTH     1     TOTAL S1TOTAL  582   2.65  4.10  3.90   28
## 3      1 2012 SOUTH     1     TOTAL S1TOTAL 3111    1.5  1.70  0.85   17
## 4      1 2012 SOUTH     1     TOTAL S1TOTAL 3112   2.01  1.80  1.60   12
## 5      1 2012 SOUTH     1     TOTAL S1TOTAL 3113   1.75  1.84  1.42   13
## 6      1 2012 SOUTH     1     TOTAL S1TOTAL 3114   1.65  1.62  0.85   15
##   FLOWERS BUDS FRUITS ANT
## 1       0    0     10  CS
## 2       0    0    150  TP
## 3       2    1     50  TP
## 4       0    0     75  CS
## 5       0    0     20  CS
## 6       0    0      0   E
\end{verbatim}

\begin{Shaded}
\begin{Highlighting}[]
\FunctionTok{str}\NormalTok{(acacia)}
\end{Highlighting}
\end{Shaded}

\begin{verbatim}
## 'data.frame':    157 obs. of  15 variables:
##  $ SURVEY   : int  1 1 1 1 1 1 1 1 1 1 ...
##  $ YEAR     : int  2012 2012 2012 2012 2012 2012 2012 2012 2012 2012 ...
##  $ SITE     : chr  "SOUTH" "SOUTH" "SOUTH" "SOUTH" ...
##  $ BLOCK    : int  1 1 1 1 1 1 1 1 1 1 ...
##  $ TREATMENT: chr  "TOTAL" "TOTAL" "TOTAL" "TOTAL" ...
##  $ PLOT     : chr  "S1TOTAL" "S1TOTAL" "S1TOTAL" "S1TOTAL" ...
##  $ ID       : int  581 582 3111 3112 3113 3114 3115 3199 941 942 ...
##  $ HEIGHT   : chr  "2.25" "2.65" "1.5" "2.01" ...
##  $ AXIS1    : num  2.75 4.1 1.7 1.8 1.84 1.62 1.95 2 2.15 5.55 ...
##  $ AXIS2    : num  2.15 3.9 0.85 1.6 1.42 0.85 0.9 1.75 1.82 4.82 ...
##  $ CIRC     : num  20 28 17 12 13 15 9 12.2 13 35 ...
##  $ FLOWERS  : int  0 0 2 0 0 0 0 0 0 0 ...
##  $ BUDS     : int  0 0 1 0 0 0 0 0 0 0 ...
##  $ FRUITS   : int  10 150 50 75 20 0 0 25 0 50 ...
##  $ ANT      : chr  "CS" "TP" "TP" "CS" ...
\end{verbatim}

\begin{Shaded}
\begin{Highlighting}[]
\CommentTok{\#summary(acacia)}
\end{Highlighting}
\end{Shaded}

After using the \texttt{str()} function, we see some small oddities in
our data set. \emph{Why is height a character vector?} This can be an
issue for some of the mathematical calculations.

Let's try and find where the issue lies. For Data Frames, we sort by
Row, Then Column. By using \texttt{df{[}Row,Column{]}} If you run
\texttt{df{[},X{]}}, you will get the full list of values from a Column.

We can also use \texttt{\$} as to sort through columns.

\begin{Shaded}
\begin{Highlighting}[]
\NormalTok{acacia[}\DecValTok{8}\NormalTok{,}\DecValTok{2}\NormalTok{] }
\end{Highlighting}
\end{Shaded}

\begin{verbatim}
## [1] 2012
\end{verbatim}

\begin{Shaded}
\begin{Highlighting}[]
\DocumentationTok{\#\#}
\NormalTok{acacia[}\DecValTok{8}\NormalTok{]}
\end{Highlighting}
\end{Shaded}

\begin{verbatim}
##     HEIGHT
## 1     2.25
## 2     2.65
## 3      1.5
## 4     2.01
## 5     1.75
## 6     1.65
## 7      1.2
## 8     1.45
## 9     1.87
## 10    2.38
## 11    2.58
## 12    2.65
## 13    2.35
## 14    1.88
## 15    2.32
## 16    2.39
## 17     2.2
## 18    1.05
## 19       2
## 20    1.28
## 21    dead
## 22     1.4
## 23     1.9
## 24    1.75
## 25     1.8
## 26     2.7
## 27    2.02
## 28     1.9
## 29    1.85
## 30    1.65
## 31     1.4
## 32     2.5
## 33    2.05
## 34    2.26
## 35    2.13
## 36     1.8
## 37    1.85
## 38     1.5
## 39    1.87
## 40    1.58
## 41    2.05
## 42    1.75
## 43    1.49
## 44    1.28
## 45    1.49
## 46    1.07
## 47    1.48
## 48    1.25
## 49    1.41
## 50     1.6
## 51     1.2
## 52    1.49
## 53     1.5
## 54    1.65
## 55    1.13
## 56    1.25
## 57     1.1
## 58     2.2
## 59    1.45
## 60     1.6
## 61    1.55
## 62     1.5
## 63    1.03
## 64    2.14
## 65     1.2
## 66    1.05
## 67     1.8
## 68     1.2
## 69    1.75
## 70    1.45
## 71    1.17
## 72    2.15
## 73     1.7
## 74    1.98
## 75    1.26
## 76    1.11
## 77    1.14
## 78    1.26
## 79     1.3
## 80    1.29
## 81    1.31
## 82    1.15
## 83    1.87
## 84    1.47
## 85    1.05
## 86     2.1
## 87    1.99
## 88    1.42
## 89     1.5
## 90    1.06
## 91    1.49
## 92     1.8
## 93    1.93
## 94     1.2
## 95    1.65
## 96    1.52
## 97    1.43
## 98    1.25
## 99    1.88
## 100   1.03
## 101    1.1
## 102    1.4
## 103   1.05
## 104   1.18
## 105    1.4
## 106   1.37
## 107   1.32
## 108   1.55
## 109    1.3
## 110   1.24
## 111    1.5
## 112   1.65
## 113   2.17
## 114   1.28
## 115   1.07
## 116   0.67
## 117   0.68
## 118   1.87
## 119   1.35
## 120   1.75
## 121   1.75
## 122   1.64
## 123   1.42
## 124   dead
## 125    0.9
## 126   dead
## 127    1.8
## 128   2.47
## 129   2.15
## 130    1.7
## 131    1.9
## 132   1.95
## 133    1.8
## 134    1.4
## 135      1
## 136   1.75
## 137   1.28
## 138      1
## 139   1.45
## 140      1
## 141   1.03
## 142   1.51
## 143   1.17
## 144   1.33
## 145    1.3
## 146   1.13
## 147   1.58
## 148   1.06
## 149   1.05
## 150   1.45
## 151   1.15
## 152   1.42
## 153   1.02
## 154    1.4
## 155   1.45
## 156   1.95
## 157   dead
\end{verbatim}

\begin{Shaded}
\begin{Highlighting}[]
\DocumentationTok{\#\#}
\NormalTok{acacia}\SpecialCharTok{$}\NormalTok{HEIGHT}
\end{Highlighting}
\end{Shaded}

\begin{verbatim}
##   [1] "2.25" "2.65" "1.5"  "2.01" "1.75" "1.65" "1.2"  "1.45" "1.87" "2.38"
##  [11] "2.58" "2.65" "2.35" "1.88" "2.32" "2.39" "2.2"  "1.05" "2"    "1.28"
##  [21] "dead" "1.4"  "1.9"  "1.75" "1.8"  "2.7"  "2.02" "1.9"  "1.85" "1.65"
##  [31] "1.4"  "2.5"  "2.05" "2.26" "2.13" "1.8"  "1.85" "1.5"  "1.87" "1.58"
##  [41] "2.05" "1.75" "1.49" "1.28" "1.49" "1.07" "1.48" "1.25" "1.41" "1.6" 
##  [51] "1.2"  "1.49" "1.5"  "1.65" "1.13" "1.25" "1.1"  "2.2"  "1.45" "1.6" 
##  [61] "1.55" "1.5"  "1.03" "2.14" "1.2"  "1.05" "1.8"  "1.2"  "1.75" "1.45"
##  [71] "1.17" "2.15" "1.7"  "1.98" "1.26" "1.11" "1.14" "1.26" "1.3"  "1.29"
##  [81] "1.31" "1.15" "1.87" "1.47" "1.05" "2.1"  "1.99" "1.42" "1.5"  "1.06"
##  [91] "1.49" "1.8"  "1.93" "1.2"  "1.65" "1.52" "1.43" "1.25" "1.88" "1.03"
## [101] "1.1"  "1.4"  "1.05" "1.18" "1.4"  "1.37" "1.32" "1.55" "1.3"  "1.24"
## [111] "1.5"  "1.65" "2.17" "1.28" "1.07" "0.67" "0.68" "1.87" "1.35" "1.75"
## [121] "1.75" "1.64" "1.42" "dead" "0.9"  "dead" "1.8"  "2.47" "2.15" "1.7" 
## [131] "1.9"  "1.95" "1.8"  "1.4"  "1"    "1.75" "1.28" "1"    "1.45" "1"   
## [141] "1.03" "1.51" "1.17" "1.33" "1.3"  "1.13" "1.58" "1.06" "1.05" "1.45"
## [151] "1.15" "1.42" "1.02" "1.4"  "1.45" "1.95" "dead"
\end{verbatim}

\begin{Shaded}
\begin{Highlighting}[]
\DocumentationTok{\#\#}
\end{Highlighting}
\end{Shaded}

There are values in \emph{HEIGHT} that we cannot keep. This is stopping
us from treating HEIGHT as an integer. Let's replace those values.

\begin{Shaded}
\begin{Highlighting}[]
\NormalTok{col\_height }\OtherTok{\textless{}{-}} \FunctionTok{as.numeric}\NormalTok{(acacia}\SpecialCharTok{$}\NormalTok{HEIGHT)}
\end{Highlighting}
\end{Shaded}

\begin{verbatim}
## Warning: NAs introduced by coercion
\end{verbatim}

\begin{Shaded}
\begin{Highlighting}[]
\NormalTok{acacia}\SpecialCharTok{$}\NormalTok{HEIGHT }\OtherTok{\textless{}{-}}\NormalTok{ col\_height}
\end{Highlighting}
\end{Shaded}

By running this command, we have \textbf{coerced} the character strings
into NA values. This way, we can run mathematical statistics on the daat
set.

If we caught this earlier, maybe \emph{before} we imported the data set,
we could've forced NAs during import.

\begin{Shaded}
\begin{Highlighting}[]
\NormalTok{acacia }\OtherTok{\textless{}{-}} \FunctionTok{read.table}\NormalTok{(acacia\_csv, }\AttributeTok{header =} \ConstantTok{TRUE}\NormalTok{, }\AttributeTok{sep =} \StringTok{"}\SpecialCharTok{\textbackslash{}t}\StringTok{"}\NormalTok{, }\AttributeTok{na.strings =} \StringTok{"dead"}\NormalTok{)}
\end{Highlighting}
\end{Shaded}

\hypertarget{can-we-make-a-graph-from-this-data}{%
\subsection{Can we make a graph from this
data?}\label{can-we-make-a-graph-from-this-data}}

\begin{Shaded}
\begin{Highlighting}[]
\FunctionTok{plot}\NormalTok{(}\AttributeTok{x=}\NormalTok{acacia}\SpecialCharTok{$}\NormalTok{HEIGHT,}\AttributeTok{y=}\NormalTok{acacia}\SpecialCharTok{$}\NormalTok{AXIS1)}
\end{Highlighting}
\end{Shaded}

\includegraphics{jw-visualiztion-uhuru_files/figure-latex/graph step 1-1.pdf}
The \texttt{plot} function is useful, but not powerful. Its hard to add
titles, and even more difficult to change plot types. How could you ever
do a heatmap via plot? You can't. So instead we use \texttt{ggplot}.

\hypertarget{how-do-we-use-ggplot}{%
\subsection{How do we use ggplot?}\label{how-do-we-use-ggplot}}

That's a goood question.

\begin{Shaded}
\begin{Highlighting}[]
\FunctionTok{library}\NormalTok{(ggplot2)}
\FunctionTok{ggplot}\NormalTok{(}\AttributeTok{data=}\NormalTok{acacia, }\AttributeTok{mapping =} \FunctionTok{aes}\NormalTok{(}\AttributeTok{x=}\NormalTok{HEIGHT, }\AttributeTok{y=}\NormalTok{ AXIS1, }\AttributeTok{color =}\NormalTok{ TREATMENT)) }\SpecialCharTok{+}
  \FunctionTok{geom\_point}\NormalTok{() }\SpecialCharTok{+}
  \FunctionTok{labs}\NormalTok{(}\AttributeTok{x=}\StringTok{"Tree Height"}\NormalTok{, }\AttributeTok{y =} \StringTok{"Tree Axis 1"}\NormalTok{)}
\end{Highlighting}
\end{Shaded}

\begin{verbatim}
## Warning: Removed 4 rows containing missing values (`geom_point()`).
\end{verbatim}

\includegraphics{jw-visualiztion-uhuru_files/figure-latex/graph step 2-1.pdf}
Use \texttt{geom\_point()} to create a scatter plot.

What if we want to reshape our axes? We have functions to
\texttt{scale()} the data set.

\begin{Shaded}
\begin{Highlighting}[]
\FunctionTok{ggplot}\NormalTok{(}\AttributeTok{data =}\NormalTok{ acacia, }\AttributeTok{mapping =} \FunctionTok{aes}\NormalTok{(}\AttributeTok{x =}\NormalTok{ HEIGHT, }\AttributeTok{y =}\NormalTok{ FRUITS, }\AttributeTok{color =}\NormalTok{ ANT)) }\SpecialCharTok{+}
  \FunctionTok{geom\_point}\NormalTok{(}\AttributeTok{size =} \DecValTok{3}\NormalTok{, }\AttributeTok{alpha=} \FloatTok{0.5}\NormalTok{) }\SpecialCharTok{+}
\DocumentationTok{\#\# Use facet\_wrap }
  \FunctionTok{facet\_wrap}\NormalTok{(}\SpecialCharTok{\textasciitilde{}}\NormalTok{ANT, }\AttributeTok{scales =} \StringTok{"free"}\NormalTok{)}
\end{Highlighting}
\end{Shaded}

\begin{verbatim}
## Warning: Removed 4 rows containing missing values (`geom_point()`).
\end{verbatim}

\includegraphics{jw-visualiztion-uhuru_files/figure-latex/graph step 3-1.pdf}

\begin{Shaded}
\begin{Highlighting}[]
\DocumentationTok{\#\#Add in geom\_smooth() for data insight\#\#}
  
\FunctionTok{ggplot}\NormalTok{(}\AttributeTok{data =}\NormalTok{ acacia, }\AttributeTok{mapping =} \FunctionTok{aes}\NormalTok{(}\AttributeTok{x =}\NormalTok{ CIRC, }\AttributeTok{y =}\NormalTok{ HEIGHT, }\AttributeTok{color =}\NormalTok{ TREATMENT)) }\SpecialCharTok{+}
  \DocumentationTok{\#\#alpha is a modifier of point transparaceny}
  \FunctionTok{geom\_point}\NormalTok{(}\AttributeTok{size =} \DecValTok{3}\NormalTok{, }\AttributeTok{alpha =} \FloatTok{0.667}\NormalTok{) }\SpecialCharTok{+} 
  \FunctionTok{geom\_smooth}\NormalTok{() }\CommentTok{\#method = " +}
\end{Highlighting}
\end{Shaded}

\begin{verbatim}
## `geom_smooth()` using method = 'loess' and formula = 'y ~ x'
\end{verbatim}

\begin{verbatim}
## Warning: Removed 4 rows containing non-finite values (`stat_smooth()`).
## Removed 4 rows containing missing values (`geom_point()`).
\end{verbatim}

\includegraphics{jw-visualiztion-uhuru_files/figure-latex/graph step 3-2.pdf}

\begin{Shaded}
\begin{Highlighting}[]
  \FunctionTok{ggsave}\NormalTok{(}\AttributeTok{filename =} \StringTok{"jw\_acacia\_treatment.jpg"}\NormalTok{)}
\end{Highlighting}
\end{Shaded}

\begin{verbatim}
## Saving 6.5 x 4.5 in image
## `geom_smooth()` using method = 'loess' and formula = 'y ~ x'
\end{verbatim}

\begin{verbatim}
## Warning: Removed 4 rows containing non-finite values (`stat_smooth()`).
## Removed 4 rows containing missing values (`geom_point()`).
\end{verbatim}

You must always call \texttt{ggplot} to access any of its graphical
interface. The differences lie in the interlayed functions.

\begin{Shaded}
\begin{Highlighting}[]
\FunctionTok{ggplot}\NormalTok{(}\AttributeTok{data=}\NormalTok{acacia, }\AttributeTok{mapping =} \FunctionTok{aes}\NormalTok{(}\AttributeTok{x=}\NormalTok{TREATMENT)) }\SpecialCharTok{+}
  \FunctionTok{geom\_bar}\NormalTok{()}
\end{Highlighting}
\end{Shaded}

\includegraphics{jw-visualiztion-uhuru_files/figure-latex/graphs3_4-1.pdf}

\begin{Shaded}
\begin{Highlighting}[]
\FunctionTok{ggplot}\NormalTok{(acacia, }\FunctionTok{aes}\NormalTok{(}\AttributeTok{x=}\NormalTok{CIRC, }\AttributeTok{color=}\NormalTok{ TREATMENT)) }\SpecialCharTok{+}
  \DocumentationTok{\#\# \textasciigrave{}bins = \textasciigrave{} defines how many boxes are displayed }
  \DocumentationTok{\#\# \textasciigrave{}Fill = \textasciigrave{} is for color }
  \FunctionTok{geom\_histogram}\NormalTok{(}\AttributeTok{bins =} \DecValTok{20}\NormalTok{, }\AttributeTok{fill =} \StringTok{"slategray3"}\NormalTok{)}
\end{Highlighting}
\end{Shaded}

\begin{verbatim}
## Warning: Removed 4 rows containing non-finite values (`stat_bin()`).
\end{verbatim}

\includegraphics{jw-visualiztion-uhuru_files/figure-latex/unnamed-chunk-4-1.pdf}

\begin{Shaded}
\begin{Highlighting}[]
  \CommentTok{\#ggsave("../197{-}figures/acacia\_Circ\_by\_treatment.jpg")}
\end{Highlighting}
\end{Shaded}


\end{document}
